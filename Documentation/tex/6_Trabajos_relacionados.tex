\capitulo{6}{Trabajos relacionados}

Para comenzar este apartado, hay que exponer que existen o podrían existir tantos sistemas de información, como empresas u organizaciones encontramos alrededor del mundo. 
Esto es así ya que cada empresa u organización posee una estructura y características muy diferenciadas que la hacen única.

Por lo tanto, hay que destacar que existen numerosos trabajos y libros sobre sistemas de información, que explican conceptos teóricos, así como la forma de implantar estos sistemas.  

Se ha encontrado numerosa documentación sobre sistemas de información geográfica y para el ámbito administrativo. Un libro de carácter teórico, donde se explican características, categorías y el funcionamiento de los diferentes tipos de sistemas sería el siguiente \emph{Introducción a la gestión de sistemas de información en la empresa}\cite{trabajos}.


También existe bastante documentación y trabajos sobre principios de los sistemas de información donde se explica el uso, desarrollo e implantación de los sistemas de información orientados al negocio\cite{effy}.

Como conclusión de este apartado, destacar que no se han encontrado trabajos de investigación muy similares al proyecto realizado, si bien es cierto que como se ha comentado, existen numerosos desarrollos de otros tipos de sistemas u orientados a otro tipo de soluciones.