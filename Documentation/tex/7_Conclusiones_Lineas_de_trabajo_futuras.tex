\capitulo{7}{Conclusiones y Líneas de trabajo futuras}

En el último apartado se van a exponer las conclusiones obtenidas tras la realización del proyecto. Adicionalmente se propondrán unas pautas o sugerencias para mejoras futuras  del proyecto.

\section{Conclusiones}\label{conclusiones}

Tras la finalización del proyecto, se extraen las siguientes conclusiones:

\begin{itemize}
\item En primer lugar hay que destacar que se han conseguido los objetivos (general y técnicos) marcados al inicio del proyecto. De esta manera, se ha logrado crear un sistema de información sencillo, útil y fiable, que se espera que sea valioso para su uso en el ámbito de la Universidad de Burgos. 

\item Gracias a la realización de este proyecto, la Universidad de Burgos posee un sistema de información idóneo para generar gráficos capaces de aportar información visual muy eficaz, pudiendo ser de gran ayuda en la toma de decisiones o detección de problemas relacionados con la matriculación de alumnos. 

\item Otro aspecto relevante es la facilidad de elección, personalización y generación de los diferentes tipos de gráficos, ya que únicamente con cuatro clics se obtienen los gráficos que se deseen.  

\item Por último, una conclusión personal. Me gustaría destacar que gracias a la realización de este proyecto, realmente se han encajado o asimilado una gran parte de las asignaturas cursadas en la carrera, ya que un pequeño proyecto como es el caso, conlleva numerosas tareas (planificación, diseño, investigación, desarrollo, documentación, entre otras muchas) y hasta el momento no se habían cohesionado todas para lograr un mismo fin. Por último destacar que los problemas encontrados durante el desarrollo, han sido uno de los principales focos de aprendizaje, sirviendo para la búsqueda de soluciones y autoaprendizaje continuo que es tan importante en esta carrera.

\end{itemize}



\section{Líneas de trabajo futuras}\label{lineas_de_trabajo_futuras}

Hay que señalar que aunque se realice la entrega del proyecto, no quiere decir que se trate de un proyecto perfecto y cerrado, ya que existen aspectos que se pueden mejorar de cara al futuro. A continuación numero una serie de cambios que me hubiera gustado realizar y los cuales se podrían realizar más adelante:


\begin{itemize}

\item Modificar la aplicación para que cuente con una interfaz gráfica más óptima visualmente. Se podría realizar mediante el uso de otras herramientas, ya que como se ha explicado la librería \emph{tkinter} de \emph{Python} tiene ciertas limitaciones.

\item Dentro de la BBDD, existen campos de las tablas, cuya información no se ha explotado o utilizado. Sería muy interesante generar otros tipos de gráficos, por ejemplo relacionados con los profesores, ya que se dispone de la estructura e información necesaria.
 
\end{itemize}