\apendice{Plan de Proyecto Software}

\section{Introducción}
Una de las fases más destacadas e imprescindible de un proyecto es la planificación. En esta fase se fijan los requisitos y se estima el tiempo y dinero que va a suponer la realización del proyecto.
Para esto, es necesario tener una idea global y a la vez concisa del proyecto que se va a realizar; de manera que ambas partes que forman el proyecto estén de acuerdo.

Así pues, vamos a dividir esta fase en dos apartados:

\begin{itemize}
\item
\textbf{Planificación temporal:} en este primer apartado se realizará una estimación de los tiempos esperados, así como fijar la fecha de inicio y fecha de fin del proyecto.
\item
\textbf{Estudio de viabilidad:} en este apartado se realizará un estudio de la viabilidad, es decir, ser capaces de apreciar si el proyecto ha sido exitoso o por el contrario ha sido un fracaso. 
\end{itemize}

Dentro de este último apartado se diferenciarán dos subapartados:

\begin{itemize}
\item
\textbf{Viabilidad económica:} se estimarán los costes y beneficios del proyecto.
\item
\textbf{Viabilidad legal:} se estudiarán las regulaciones legales que pudieran afectar al proyecto.
\end{itemize}


\section{Planificación temporal}
Para realizar una planificación correcta del proyecto, se decidió utilizar una metodología ágil de desarrollo, para ello se utilizó la metodología \emph{Scrum}. Como se explica en la memoria:

\begin{itemize}
	\item Se ha utilizado una estrategia orientada a un desarrollo incremental y basada en \emph{sprints}.
	\item La duración media de cada \emph{sprint} era aproximadamente de una semana.
	\item Al inicio de cada \emph{sprint} se definían las tareas o \emph{issues} a realizar, las cuales tenían que ser realizadas en un cierto intervalo de tiempo.
	\item Cada \emph{sprint} se planificaba cuando se finalizaban las tareas o \emph{issues} del anterior \emph{sprint}.	
	\item Al final de cada \emph{sprint} se revisan todas las tareas realizadas, así como ver si se han logrado los objetivos fijados y solucionado los problemas encontrados.
\end{itemize}

A continuación, se van a detallar los \emph{sprints} que se han realizado a lo largo del proyecto:

\subsection{Sprint 0 (25/02/19-10/03/19)}
En la primera reunión de planificación de proyecto se desarrollaron y expusieron las ideas del mismo. Además se propusieron las siguientes tareas:

\begin{itemize}
\item
\textbf{Crear repositorio de HitHub.} Además de crear el repositorio principal, se solicitó y posteriormente se adquirió \emph{Github Student Developer Pack}.
\item
\textbf{Descargar plantilla para la documentación}
\item
\textbf{Redactar los objetivos del proyecto}
\item
\textbf{Probar a leer archivos Excel con Python}  
\end{itemize}

Se estiman 12 horas de trabajo.


\subsection{Sprint 1 (11/03/19-17/03/19)}
En la segunda reunión se propusieron las siguientes tareas:

\begin{itemize}
\item
\textbf{Crear un algoritmo o proceso capaz de leer los archivos Excel.} En una tarea del anterior \emph{sprint} se apreció que no se podían leer los archivos Excel con Python, ya que los archivos tenían un error de formato y extensión. Por esta razón se optó por realizar un algoritmo que fuera capaz de leer estos archivos Excel(.xls) en modo texto, y parsear todo el contenido para generar otro archivo nuevo(.csv).
\end{itemize}

Se estiman 20 horas de trabajo.


\subsection{Sprint 2 (18/03/19-24/03/19)} 
En la tercera reunión se propusieron las siguientes tareas:

\begin{itemize}
\item
\textbf{Mejorar el algoritmo para leer los archivos Excel:} mejorar la expresión regular para que no fuera tan genérica y fuera más específica.
\item
\textbf{Redactar casos de uso 1 y 2}
\item
\textbf{Eliminar apartado \emph{Objetivos personales} de la documentación}
\item
\textbf{Descargar \emph{GitHub Desktop} y \emph{gVim 8.1}}
\end{itemize}

Se estiman 18 horas de trabajo.

\subsection{Sprint 3 (25/03/19-31/03/19)}
En la cuarta reunión se propusieron las siguientes tareas:

\begin{itemize}
\item
\textbf{Cambiar la forma de parsear los datos del algoritmo.} Obtener primero la información por filas, después por celdas y por último por DATA o contenido de las celdas. Pudiendo de esta manera obtener valores como \emph{MergeDown} y \emph{MergeAcross} que aportan información necesaria sobre la separación o combinación de celdas.   
\end{itemize}

Se estiman 20 horas de trabajo.

\subsection{Sprint 4 (01/04/19-07/04/19)}
En la quinta reunión se propusieron las siguientes tareas:

\begin{itemize}
\item
\textbf{Mejorar los nombres de variables del algoritmo}
\item
\textbf{Mejorar los casos de uso 1 y 2}
\item
\textbf{Redactar caso de uso 3}
\item
\textbf{Pensar si es necesario crear una Base de Datos y cómo crearla en tal caso}
\item
\textbf{Crear gráfico apilado horizontalmente(caso de uso 2)}
\end{itemize}

Se estiman 18 horas de trabajo.

\subsection{Sprint 5 (08/04/19-25/04/19)}
Hay que indicar que al coincidir con las vacaciones de Semana Santa, este \emph{sprint} tuvo una duración superior a una semana. En la sexta reunión se propusieron las siguientes tareas:

\begin{itemize}
\item
\textbf{Empezar a crear la interfaz gráfica.} Sacar un cuadro de diálogo cuando se pulse un botón de \emph{Cargar} que permita seleccionar ficheros.
\item
\textbf{Crear modelo Entidad-Relación}
\item
\textbf{Probar a entrecomillar los \emph{strings} de los (.csv) para poder leerlos bien}
\end{itemize}

\subsection{Sprint 6}
\subsection{Sprint 7}
\subsection{Sprint 8}
\subsection{Sprint 9}
\subsection{Sprint 10}

\section{Estudio de viabilidad}

\subsection{Viabilidad económica}

\subsection{Viabilidad legal}


