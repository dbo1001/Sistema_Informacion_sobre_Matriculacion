\capitulo{5}{Aspectos relevantes del desarrollo del proyecto}

En este apartado se van a recoger los aspectos más importantes que han surgido en el desarrollo del proyecto. Se incluirán la toma de decisiones, los posibles cambios, la aparición de problemas y las soluciones establecidas.

\section{Inicio del proyecto}\label{inicio_del_proyecto}





\section{Metodologías}\label{metodologias}

A lo largo del desarrollo del proyecto se ha usado la \emph{metodología Scrum}. Se trata de una metodología ágil basada en \emph{sprints}, en este caso, de desarrollo incremental con revisiones semanales.

Por lo tanto, la duración estimada de cada \emph{sprint} es de una semana. Al finalizar cada \emph{sprint}, se planificaba el siguiente, creando sus \emph{issues} o tareas a realizar en dicho \emph{sprint}. Cuando estas tareas se realizaban, se cambiaba el estado del \emph{issue} correspondiente a \emph{Closed}.  


\section{Toma de decisiones}\label{toma_de_decisiones}


\section{Librerías para el tratamiento y manipulación de datos}\label{librerias}


\section{Interfaz de usuario del proyecto}\label{interfaz_de_usuario_del_proyecto}


\section{Problemas encontrados}
\subsection{Error al abrir los Excel(.xls) bajados de Sigma con Python}
Los archivos Excel(.xls) suministrados (descargados de plataforma \emph{Sigma}) no cumplen el estándar. Al abrirlos tanto con Excel como con varias librerías de Python, muestran un error de formato y extensión. Por lo tanto la única solución encontrada, ha sido realizar un parseo previo de los Excel suministrados, creando un fichero (.csv) nuevo, con toda la información del fichero original corrupto.

De esta manera, se ha creado un analizador sintáctico capaz de leer los ficheros originales (.xls) en modo texto y obtener un (.xml). Se ha parseado toda la información obteniendo filas, celdas, separaciones entre las mismas, contenidos de cada celda...etc. A la vez que se extrae toda esta información, se crea un fichero (.csv) nuevo y se van introduciendo los datos.

