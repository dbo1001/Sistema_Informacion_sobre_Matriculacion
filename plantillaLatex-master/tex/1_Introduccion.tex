\capitulo{1}{Introducción}

%Descripción del contenido del trabajo y del estrucutra de la memoria y del resto de materiales entregados.

En la actualidad, existe una gran cantidad de información o datos, los cuales componen una parte muy importante en las grandes  empresas y organizaciones de todo el mundo.
Cada día se genera multitud de nueva información y es indispensable almacenarla para posteriormente poder interpretarla correctamente. Es imprescindible por lo tanto, saber extraer e identificar información relevante a partir de ficheros o documentos poco legibles o difíciles de entender a priori.


En este punto es cuando toma especial interés la creación de un Sistema de Información, o lo que es lo mismo, un almacén electrónico. En dichos almacenes se protege y mantiene una gran cantidad de datos e información, de manera fiable, segura y fácil de administrar.

Además de estas funciones de almacenamiento y administración, un Sistema de Información también permite organizar, entender y utilizar los datos para la toma de decisiones.
Para esta tarea, es necesario contar con cierta capacidad de análisis, ya que hay que extraer información concreta, destacada y relevante; para posteriormente poder visualizarla con ayuda de elementos visuales como gráficos.

En la realización de este proyecto se propone la creación de un Sistema de Información, para procesar, almacenar y representar visualmente la información sobre la matriculación de alumnos en la Universidad de Burgos. 



\section{Estructura de la memoria}\label{estructura-de-la-memoria}
La memoria se estructura de la siguiente manera:

\begin{itemize}
\item
\textbf{Introducción:} se describe brevemente el contexo y el proyecto realizado. Posteriormente se realiza una sección donde se expone la estructura de la memoria.
\item
\textbf{Objetivos del proyecto:} se exponen los objetivos del proyecto, divididos en objetivo general y objetivos técnicos.
%\item
%\textbf{Conceptos teóricos:} se exponen los conceptos básicos para comprender tanto el proyecto como el desarrollo del mismo.
%\item
%\textbf{Técnicas y herramientas:} se explican las metodologías y herramientas utilizadas durante el desarrollo del proyecto.
%\item
%\textbf{Trabajos relacionados:} se exponen aplicaciones, proyectos y empresas que ofrecen soluciones en un campo similar al caso de estudio.
%\item
%\textbf{Conclusiones y líneas de trabajo futuras:} se explican las conclusiones finales que se obtienen después de la realización del proyecto, así como futuras mejoras.
\end{itemize}