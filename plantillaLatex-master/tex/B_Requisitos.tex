\apendice{Especificación de Requisitos}

\section{Introducción}

En este anexo se va a realizar y formalizar la especificación de requisitos que define el comportamiento del sistema desarrollado en el proyecto.

\section{Objetivos generales}

Los objetivos generales que se persiguen en este proyecto son los siguientes:

\begin{itemize}
\item
Desarrollar una herramienta que permita la extracción, tratamiento y análisis de datos relacionados con la matriculación de alumnos en la Universidad de Burgos (UBU). 
\item
Realizar gráficos y estadísticos que resulten visuales y aporten información valiosa al usuario final. 
\end{itemize}


\section{Catalogo de requisitos}

En este apartado se van a enumerar los requisitos específicos derivados de los objetivos proyecto, divididos en funcionales y no funcionales.

\subsection{Requisitos funcionales}

\begin{itemize}
	\item \textbf{RF-1: Importar datos:} La aplicación debe ser capaz de importar los datos desde ficheros Excel(.xls) cuyo formato y extensión de archivo no coinciden.
	\begin{itemize}
		\item \textbf{RF-1.1} Pasar datos de fichero original (.xls) a un fichero (.csv): se realizará un parseo de datos.
		\item \textbf{RF-1.2} Pasar datos de fichero (.csv) a la Base de Datos.
	\end{itemize}
	
	\item \textbf{RF-2: Gráfico apilado de Asignaturas por Curso o Semestre:} La aplicación debe ser capaz de mostrar un gráfico apilado cuyo \emph{eje x} muestre las diferentes asignaturas y su \emph{eje y} muestre la cantidad total de alumnos matriculados en las mismas. Al tratarse de un gráfico apilado, se podrán diferenciar los grupos existentes en las asignaturas(grupo online, grupo presencial 1, grupo presencial 2...etc).
	
	\item \textbf{RF-3: Gráfico de máximos, mínimos y medias por curso:} La aplicación debe ser capaz de mostrar un gráfico cuyo \emph{eje x} muestre los diferentes cursos(1º, 2º, 3º y 4º) y su \emph{eje y} muestre la cantidad total de alumnos matriculados, indicando los máximos, mínimos y medias por cada curso.

\end{itemize}


\subsection{Requisitos no funcionales}

\begin{itemize}
\item
\textbf{RNF-1: Usabilidad:} La herramienta debe ser intuitiva y fácil de utilizar, así como tener una una estructura clara y contar con una interfaz amigable.
\item
\textbf{RNF-2: Escalabilidad:} La herramienta debe permitir la incorporación de nuevas funcionalidades o nuevos módulos. 
\item
\textbf{RNF-3: Rendimiento:} La herramienta debe funcionar de forma fluida sin que la interfaz gráfica se quede bloqueada.
\item
\textbf{RNF-4: Fiabilidad:} La herramienta debe ser segura y debe funcionar correctamente bajo determinadas condiciones.  
\item
\textbf{RNF-4: Integridad:} La herramienta debe cumplir con la integración de datos y no tener pérdidas de información, así como modificación de la misma. 

\end{itemize}
\newpage


\section{Especificación de requisitos}

En este apartado se va a visualizar a través de diagramas los casos de uso de los requisitos funcionales previamente definidos. Para esto, se va a utilizar la notación UML.

\subsection{Actores}

El actor del sistema es la persona que maneja la aplicación.

\subsection{Casos de uso}




\tablaSmallSinColores{Caso de uso 1: Importar Datos }{p{3cm} p{.75cm} p{9.5cm}}{b1}{
\multicolumn{3}{l}{Caso de uso 1: Importar Datos} \\
}
{
Descripción & \multicolumn{2}{p{10.25cm}}{La aplicación debe ser capaz de importar los datos desde ficheros Excel(.xls) cuyo formato y extensión de archivo no coinciden.} \\\hline
\multirow{2}{3.5cm}{Requisitos} 
&\multicolumn{2}{p{10.25cm}}{RF-1} 
\\\cline{2-3}
& \multicolumn{2}{p{10.25cm}}{RF-1.1}
\\\cline{2-3}
& \multicolumn{2}{p{10.25cm}}{RF-1.2}
\\\cline{2-3}
Precondiciones & \multicolumn{2}{p{10.25cm}} {Existe un fichero Excel con datos para importar. Dicho fichero se ha descargado de \emph{Sigma} y se encuentra en una carpeta local.}
\\\hline
\multirow{2}{3.5cm}{Secuencia normal} & Paso & Acción \\\cline{2-3}
& 1 & El usuario pulsa el botón de cargar archivos.
\\\cline{2-3}
& 2 & Se muestra un desplegable para poder navegar por el directorio y seleccionar los ficheros disponibles.
\\\cline{2-3}
& 3 & Se crea un fichero (.csv) a partir del fichero(.xls) para poder tratar la información.
\\\cline{2-3}
& 4 & Se graba el archivo (.csv).
\\\cline{2-3}
& 5 & Se guardan los datos en la Base de Datos.
\\\hline
Postcondiciones & \multicolumn{2}{p{10.25cm}}{Se crea un fichero (.csv) nuevo y los datos se cargan en la Base de Datos.} \\\hline
Excepciones & \multicolumn{2}{p{10.25cm}}{Si los datos se encontraran en un fichero (.csv) el RF-1.1 no sería necesario.}\\\hline
Importancia & Alta \\\hline
Urgencia & Alta \\\hline
Comentarios & & Los ficheros Excel no cumplen el estándar(al abrirlos muestra un error de formato y extensión) y es necesario hacer un filtrado previo.  \\
}


\tablaSmallSinColores{Caso de uso 2: Gráfico apilado de Asignaturas por Curso }{p{3cm} p{.75cm} p{9.5cm}}{b1}{
\multicolumn{3}{l}{Caso de uso 2: Gráfico apilado de Asignaturas por Curso} \\
}
{
Descripción & \multicolumn{2}{p{10.25cm}}{La aplicación debe ser capaz de mostrar un gráfico apilado cuyo \emph{eje x} muestre las diferentes asignaturas y su \emph{eje y} muestre la cantidad total de alumnos matriculados en las mismas. Al tratarse de un gráfico apilado, se podrán diferenciar los grupos existentes en las asignaturas(grupo online, grupo presencial 1, grupo presencial 2...etc).} \\\hline
\multirow{2}{3.5cm}{Requisitos} 
&\multicolumn{2}{p{10.25cm}}{RF-2} 
\\\cline{2-3}
Precondiciones & \multicolumn{2}{p{10.25cm}} {Existe un fichero Excel con datos para importar. Dicho fichero se ha descargado de \emph{Sigma} y se encuentra en una carpeta local.}
\\\hline
\multirow{2}{3.5cm}{Secuencia normal} & Paso & Acción \\\cline{2-3}
& 1 & Se seleccionan los campos necesarios de la Base de Datos.
\\\cline{2-3}
& 2 & .
\\\cline{2-3}
& 3 & .
\\\hline
Postcondiciones & \multicolumn{2}{p{10.25cm}}{La aplicación debe mostrar el gráfico apilado correctamente.} \\\hline
Excepciones & \multicolumn{2}{p{10.25cm}}{}\\\hline
Importancia & Alta \\\hline
Urgencia & Alta \\\hline
Comentarios & & Con esta gráfica se puede apreciar si existe algún grupo desequilibrado, así como las asignaturas con más matriculados entre otra información destacable.   \\
}